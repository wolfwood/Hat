\documentclass[12pt]{article}

\usepackage[a4paper]{geometry}

\newif\ifpdf
\ifx\pdfoutput\undefined
  \pdffalse
\else
  \pdfoutput=1
  \pdftrue
\fi

\ifpdf
  \usepackage[colorlinks=true]{hyperref}
  \usepackage[pdftex]{graphicx}
  \pdfcompresslevel=9
\else
  \usepackage{graphicx}
\fi

%------------------------------------------------------------------------------



\newcommand{\CUT}{\mbox{\rule{1.0ex}{1.5ex}}}
\newcommand{\highlight}[1]{\raisebox{0pt}[2ex][2ex]{\fbox{\tt #1}}}
\newcommand{\mkblue}[1]{#1}
\begin{document}
\title{Hat Tutorial}
\author{The ART Team \\
{\tt ART-team@cs.york.ac.uk}}
\date{29 April 2005}
\maketitle
\section{Introduction}

This tutorial is intended as a practical introduction to the
Hat tools\footnote{
Available from {\tt http://www.haskell.org/hat/}.}
for tracing Haskell 98 programs.
It introduces the basic ideas
and explains with worked examples how to use the tools.

Readers are encouraged to follow the
tutorial using an installation of Hat.
This first version of the tutorial assumes Hat (Version 2.04), nhc98, hmake
 and Linux, but it works equally
well with ghc instead of nhc98.

There are several Hat tools for examining traces, but this tutorial will
consider only two: {\tt hat-trail} and {\tt hat-observe}.
Even for these tools not every option and command is discussed.
For a more comprehensive reference see the Hat User Manual.

The tutorial makes use of a small example program ---
at first a correctly working version, later one with faults
deliberately introduced.
The intended behaviour of the program is very
simple: it should sort the letters of the word `program' using insertion sort.
The working program is given\footnote{
The program may also be found in the file {\tt Insort.hs}.}
in Figure~\ref{insort}.

\begin{figure}[t]
\fbox{\parbox{\textwidth}{
\begin{tabular}{lcl}
\multicolumn{3}{l}{\tt sort ::\ Ord a => [a] -> [a]} \\
{\tt sort []}     & {\tt =} & {\tt []} \\
{\tt sort (x:xs)} & {\tt =} & {\tt insert x (sort xs)}
\end{tabular} \\
\begin{tabular}{lcl}
\multicolumn{3}{l}{\tt insert ::\ Ord a => a -> [a] -> [a]} \\
{\tt insert x []}     & {\tt =} & {\tt [x]} \\
{\tt insert x (y:ys)} & {\tt =} & {\tt if x <= y then x :\ y :\ ys} \\
{\tt }                & {\tt  } & {\tt else y :\ insert x ys}
\end{tabular} \\
\begin{tabular}{lcl}
{\tt main} & {\tt =} & {\tt putStrLn (sort "program")}
\end{tabular}
}}
\caption{\label{insort}An insertion-sort program.}
\end{figure}

\section{Hat Compilation and Execution}

To use Hat, the Haskell program to be traced must first be
compiled with the {\tt -hat} option to {\tt hmake}:

\begin{tabbing}
{\tt \$ hmake -hat Insort} \\
{\tt hat-trans Insort.hs} \\
{\tt Creating directories Hat} \\
{\tt Wrote Hat/Insert.hs} \\
{\tt nhc98 -package hat -c -o Hat/Insort.o Hat/Insort.hs} \\
{\tt nhc98 -package hat -o Insort Hat/Insort.o}
\end{tabbing}

A program compiled for tracing can be executed just as
if it had been compiled normally.
\begin{tabbing}
{\tt \$ Insort} \\
{\tt agmoprr}
\end{tabbing}
The main difference from untraced execution is that as {\tt Insort} runs
it records a detailed trace of its computation in a file {\tt Insort.hat}.
The trace is a graph of program expressions encoded in a special-purpose
binary format.

Two further files {\tt Insort.hat.output} and
{\tt Insort.hat.bridge}
record the output and associated references to the trace file.\footnote{
Trace files do not include program sources, but they do include
references to program sources; modifying source files could invalidate
source links from traces.}

\section{Hat-trail: Basics}

After a program compiled for tracing has been run, creating a trace file,
special-purpose tools are used to examine the trace.
The first such tool we shall look at is hat-trail.
The idea of hat-trail is to answer the question `Where did that come
from?' in relation to values, expressions, outputs and error messages.
The immediate answer will be a parent application or name.
More specifically:
\begin{itemize}
\item {\em errors}:
the application or name being reduced when the error occurred
(eg. {\tt head []} might be the parent of a pattern-match failure);

\item {\em outputs}:
the monadic action that caused the output
(eg. {\tt putStr "Hello world"} might the parent of a section of output text);
 
\item {\em non-value expressions}:
the application or name whose defining body
contains the expression of which the child is an instance
(eg. {\tt append [1,2] [3,4]} might be the parent of {\tt append [2] [3,4]});
 
\item {\em values}:
as for non-value expressions, or the application of a
predefined function with the child as result
(eg. {\tt [1,2]++[3,4]} might be the parent of {\tt [1,2,3,4]}).
\end{itemize}
Parent expressions, and their subexpressions, may in turn have parents
of their own.
The tool is called hat-trail because it
displays trails of ancestral redexes, tracing effects back to their
causes.

\subsubsection*{Hat-trail sessions and requests}
A hat-trail session can be started from a shell command line, or from
within existing sessions of hat tools.
The immediate result of the shell command
\begin{tabbing}
{\tt \$ hat-trail Insort}
\end{tabbing}
is the display of a terminal window with an upper part headed {\tt Output}
and a lower part headed {\tt Trail}:
\begin{tabbing}
{\tt Output:\ -------------------------------------------------------} \\
\highlight{agmoprr\char92n} \\
\\
{\tt Trail:\ ------ hat-trail 2.00  (:h for help, :q to quit) -------}
\end{tabbing}
The line of output is highlighted\footnote{
In the printed version of this tutorial, highlighted text or
expressions are shown boxed; the Hat tools actually use colour for
highlighting.}
because it is the current selection.

Requests in {\tt hat-trail} are of two kinds. Some are single key presses
with an immediate response; others are command-lines starting with
a colon and only acted upon when completed by keying return.
A basic repertoire of single-key requests is:
\begin{center}
\begin{tabular*}{\textwidth}{lp{10cm}}
{\em return}      & add to the trail the parent expression of
                    the current selection \\
{\em backspace}   & remove the last addition to the trail display \\
{\em arrow keys}  & select (a) parts of the output generated by
                    different actions, or (b) subexpressions of expressions
	            already on display
\end{tabular*}
\end{center}
And a basic repertoire of command-line requests is:
\begin{center}
\begin{tabular*}{\textwidth}{lp{10cm}}
{\tt :source} & show the source expression of which the current
                selection is an instance \\
{\tt :quit}   & finish this hat-trail session
\end{tabular*}
\end{center}
It is enough to give initial letters, {\tt :s} or {\tt :q}, rather
than {\tt :source} or {\tt :quit}.

\subsubsection*{Some {\tt Insort} trails}
To trace the output from the {\tt Insort} computation, keying return
alters the {\tt Trail} part of the display to:
\begin{tabbing}
{\tt Trail:\ ----------------------- Insort.hs line:\ 10 col:\ 8 ------} \\
{\tt <- }\highlight{putStrLn "agmoprr"}
\end{tabbing}
The source reference is to the corresponding application of {\tt putStrLn} in
the program.  Giving the command {\tt :s} at this point creates a separate
source window showing the relevant extract of the program.\footnote{
The only
thing to do with a source extract is to look at it: tracing with Hat
does not involve annotating or otherwise modifying program sources.}

Back to the {\tt Trail} display. Keying return again:
\begin{tabbing}
{\tt Trail:\ ----------------------- Insort.hs line:\ 10 col:\ 1 ------} \\
{\tt <-}\highlight{putStrLn "agmoprr"} \\
{\tt <-}\highlight{main}
\end{tabbing}
That is, the line of output was produced by an application of
{\tt putStrLn} occurring in the body of {\tt main}.

So far, so good; but what about the sorting?  How do we see where {\tt putStr}'s
string argument {\tt "agmoprr"} came from?  By making that string the current
selection and requesting its parent:
\begin{tabbing}
{\em right-arrow}\hspace{10mm}\=\kill
{\em backspace}   \> (removes {\tt main}), \\
{\em right-arrow} \> (selects {\tt putStrLn}), \\
{\em right-arrow} \> (selects {\tt "agmoprr"}), \\
{\em return}      \> (requests parent expression)
\end{tabbing}
\begin{tabbing}
{\tt Trail:\ ----------------------- Insort.hs line:\ 7 col:\ 19 -----} \\
{\tt <- putStrLn\ }\highlight{"agmoprr"} \\
{\tt <-\ }\highlight{insert 'p' "agmorr" | if False}
\end{tabbing}
The string {\tt "agmoprr"} is the result of inserting {\tt 'p'}, the head of the
string {\tt "program"}, into the recursively sorted tail.  More specifically,
the string was computed in the else-branch of the conditional by which
{\tt insert} is defined in the recursive case
(because {\tt 'p' <= 'a'} is {\tt False}).

And so we could continue.  For example, following the trail of
string arguments:
\begin{tabbing}
{\tt <- insert 'p'\ }\highlight{"agmorr"}{\tt \ | if False} \\
{\tt <- insert 'r'\ }\highlight{"agmor"}{\tt \ | if False} \\
{\tt <- insert 'o'\ }\highlight{"agmr"}{\tt \ | if False} \\
{\tt <- insert 'g'\ }\highlight{"amr"}{\tt \ | if False} \\
{\tt <- insert 'r'\ }\highlight{"am"}{\tt \ | if False} \\
{\tt <- insert 'a'\ }\highlight{"m"}{\tt \ | if True} \\
{\tt <-\ }\highlight{insert 'm' []}
\end{tabbing}
But let's leave hat-trail for now.
\begin{tabbing}
{\tt :quit} \\
\end{tabbing}

\section{Hat-observe: Basics}
The idea of hat-observe is to answer the question `To which arguments,
if any, was that applied, and with what results?', mainly in relation
to a top-level function.
Answers take the form of a list of equational observations, showing
for each application of the function to distinct arguments what result
was computed.  The user has the option to limit observations to particular
patterns of arguments or results, or to particular application contexts. 

\subsubsection*{Hat-observe sessions and requests}
A hat-observe session can be started from a shell command line, or from
within existing sessions of hat tools.
\begin{tabbing}
{\tt \$ hat-observe Insort} \\ \\
{\tt \ \ \ \ \ \ \ \ hat-observe 2.00  (:h for help, :q to quit)} \\ \\
{\tt hat-observe>} \\
\end{tabbing}
In comparison with hat-trail, there is more emphasis on command-lines
in hat-observe,
and the main user interface is a prompt-request-response
cycle.
Requests are of two kinds.
Some are observation queries in the
form of application patterns:
the simplest observation query is just the name of a top-level function.
Others are command-lines, starting with
a colon, similar to those of hat-trail.
A basic repertoire of command-line requests is
\begin{center}
\begin{tabular*}{\textwidth}{lp{10cm}}
{\tt :info } & list the names of functions and other defined
               values that can be observed, with application counts \\
{\tt :quit}  & finish this hat-observe session \\
\end{tabular*}
\end{center}
Again it is enough to give the initial letters, {\tt :i} or {\tt :q}.
\subsubsection*{Some {\tt Insort} observations}
A common way to begin a hat-observe session is with an {\tt :info}
request, followed by initial observation of central functions.
\begin{tabbing}
{\tt hat-observe> :info} \\
\hspace{2.5cm}\=\hspace{2.5cm}\=\hspace{2.5cm}\=\hspace{2.5cm}\=\kill
{\tt 19 <=}       \> {\tt 21 insert}        \> {\tt 1 main}   \>
{\tt 1 putStrLn}  \> {\tt 1 sort} \\
{\tt hat-observe> sort} \\
{\tt \mkblue{1} sort "program" = "agmoprr"} \\
{\tt \mkblue{2} sort "rogram" = "agmorr"} \\
{\tt \mkblue{3} sort "ogram" = "agmor"} \\
{\tt \mkblue{4} sort "gram" = "agmr"} \\
{\tt \mkblue{5} sort "ram" = "amr"} \\
{\tt \mkblue{6} sort "am" = "am"} \\
{\tt \mkblue{7} sort "m" = "m"} \\
{\tt \mkblue{8} sort [] = []}
\end{tabbing}
Here the number of observations is small.  Larger collections of
obervations are presented in blocks of ten (by default).
\begin{tabbing}
{\tt hat-observe> <=} \\
{\tt \mkblue{1} 'a' <= 'm' = True} \\
{\tt \mkblue{2} 'r' <= 'a' = False} \\
{\tt \mkblue{3} 'g' <= 'a' = False} \\
{\tt \mkblue{4} 'o' <= 'a' = False} \\
{\tt \mkblue{5} 'p' <= 'a' = False} \\
{\tt \mkblue{6} 'r' <= 'm' = False} \\
{\tt \mkblue{7} 'g' <= 'm' = True} \\
{\tt \mkblue{8} 'o' <= 'g' = False} \\
{\tt \mkblue{9} 'r' <= 'g' = False} \\
{\tt \mkblue{10} 'p' <= 'g' = False} \\
{\tt --more--> }
\end{tabbing}
Keying return in response to {\tt --more-->} requests the next block of
observations.  Alternatively, requests in the colon-command family can
be given.  Any other line of input cuts
short the list of reported observations in favour of a fresh {\tt hat-observe>}
prompt.
\begin{tabbing}
{\tt --more--> n} \\
{\tt hat-observe>}
\end{tabbing}

\subsubsection*{Observing restricted patterns of applications}
Viewing a block at a time is not the only way of handling what may be
a large number of applications.  Obervations can also be restricted to
applications in which specific patterns of values occur as arguments or result, or
to applications in a specific context.  The full syntax for obervation
queries is
\begin{tabbing}
{\tt identifier pattern* [= pattern] [in identifier]}
\end{tabbing}
where the {\tt *} indicates that there can be zero or more occurrences of
an argument pattern and the {\tt [...]} indicate that the result pattern and
context are optional.  Patterns in observation queries are simplified
versions of constructor patterns with \verb"_" as the only variable.
Some examples for the {\tt Insort} computation:
\begin{tabbing}
{\tt hat-observe> insert 'g' \verb"_"} \\
{\tt \mkblue{1} insert 'g' "amr" = "agmr"} \\
{\tt \mkblue{2} insert 'g' "mr" = "gmr"} \\
{\tt hat-observe> insert \verb"_" \verb"_" = [\verb"_"]} \\
{\tt \mkblue{1} insert 'm' [] = "m"} \\
{\tt \mkblue{2} insert 'r' [] = "r"} \\
{\tt hat-observe> sort in main} \\
{\tt \mkblue{1} sort "program" = "agmoprr"} \\
{\tt hat-observe> sort in sort} \\
{\tt \mkblue{1} sort "rogram" = "agmorr"} \\
{\tt \mkblue{2} sort "ogram" = "agmor"} \\
{\tt \mkblue{3} sort "gram" = "agmr"} \\
{\tt \mkblue{4} sort "ram" = "amr"} \\
{\tt \mkblue{5} sort "am" = "am"} \\
{\tt \mkblue{6} sort "m" = "m"} \\
{\tt \mkblue{7} sort [] = []}
\end{tabbing}
Enough on hat-observe for now.
\begin{tabbing}
{\tt hat-observe> :quit}
\end{tabbing}

\section{Tracing Faulty Programs}

We have seen so far some of the ways in which Hat tools can be used
to trace a correctly working program.  But a common and intended use
for Hat is to trace a faulty program with the aim of locating the
source of the faults.  A faulty computation has one of three outcomes:
\begin{enumerate}
\item
termination with a run-time error, or
\item
termination with incorrect output, or
\item
non-termination.
\end{enumerate}
A variant of {\tt Insort}
given\footnote{The program may also be found in the file {\tt BadInsort.hs}.}
in Figure~\ref{badinsort} has three faults,
each of which alone
would cause a different outcome, as indicated by comments.
\begin{figure}[t]
\fbox{\parbox{\textwidth}{
\begin{tabular}{lcl}
\multicolumn{3}{l}{\tt sort ::\ Ord a => [a] -> [a]} \\
\multicolumn{3}{l}{\tt -- FAULT (1):\ missing equation for [] argument} \\
{\tt sort (x:xs)} & {\tt =} & {\tt insert x (sort xs)}
\end{tabular} \\
\begin{tabular}{lcl}
\multicolumn{3}{l}{\tt insert ::\ Ord a => a -> [a] -> [a]} \\
{\tt insert x []}     & {\tt =} & {\tt [x]} \\
{\tt insert x (y:ys)} & {\tt =} & {\tt if x <= y} \\
                      &         & {\tt -- FAULT (2):\ y missing from result} \\
                      &         & {\tt then x : ys} \\
                      &         & {\tt -- FAULT (3):\ recursive call is same} \\
                      &         & {\tt else y :\ insert x (y:ys)}
\end{tabular} \\
\begin{tabular}{lcl}
{\tt main} & {\tt =} & {\tt putStrLn (sort "program")}
\end{tabular}
}}
\caption{\label{badinsort}A faulty version of the insertion-sort program.}
\end{figure}
In the following sections we shall apply the Hat tools to examine the
faulty program, as if we didn't know in advance where the faults were.

\subsection{Tracing a Run-time Error}
We compile the faulty program for tracing, then run it:
\begin{tabbing}
{\tt \$ hmake -hat BadInsort} \\
{\tt ...} \\
{\tt \$ BadInsort } \\
{\tt No match in pattern. }
\end{tabbing}
Two questions prompted by this error message are:
\begin{itemize}
\item
What was the application that didn't match?
\item
Where did that application come from?
\end{itemize}

\subsubsection*{Using hat-trail}

Both questions can be answered by using hat-trail to trace the derivation
of the error. 
\begin{tabbing}
{\tt \$ hat-trail BadInsort} \\
\\
{\tt Error:\ --------------------------------------------------------} \\
\highlight{No match in pattern.} \\
\end{tabbing}
Keying return once to see the erroneous application,
then again to see its parent application:
\begin{tabbing}
{\tt Trail:\ ------------------- BadInsort.hs line:\ 3 col:\ 25 -------} \\
{\tt <-\ }\highlight{sort []} \\
{\tt <-\ }\highlight{sort "m"}
\end{tabbing}
This information can be supplemented by reference to the source program.
With {\tt sort []} selected, the {\tt :source} command
shows the site of the offending application in the
recursive equation for {\tt sort}.
If necessary the ancestry of the {\tt []} argument or
the {\tt sort} application could be traced back further\footnote{
In general, when
tracing the origins of an application there are five choices: trace the
ancestry of the function, or of an argument, or of the application itself;
alternatively look at the source context of the application, or at the
source definition of the function}.
% For example, selecting
% the {\tt []} argument (right-arrow twice) shows its origin to be the string
% literal {\tt "program"} in {\tt main}.

\subsubsection*{Using hat-observe}

Although hat-trail is usually the first resort for tracing run-time
errors, it is instructive to see what happens if instead we try
using hat-observe.
\begin{tabbing}
{\tt \$ hat-observe BadInsort } \\ \\
{\tt \ \ \ \ \ \ \ \ hat-observe 2.00  (:h for help, :q to quit)} \\ \\
\hspace{2.5cm}\=\hspace{2.5cm}\=\hspace{2.5cm}\=\hspace{2.5cm}\=\kill
{\tt hat-observe> :info } \\
{\tt 7+0 insert} \> {\tt 1 main} \> {\tt 1 putStrLn} \> {\tt 1+7 sort}
\end{tabbing}
What do the {\it M}{\tt +}{\it N} counts for {\tt insert} and {\tt sort}
mean?  {\it M} is the number of applications that never got
beyond a pattern-matching stage involving evaluation of arguments;
{\it N} is the number of applications that were actually reduced
to an instance of the function body.
Applications are only counted at all if their results were
demanded during the computation.
Where a count is shown as a single number, it is a count of applications
actually reduced.

In the {\tt BadInsort} computation, we see 
there are fewer obervations of {\tt insert} than there were
in the correct {\tt Insort}
computation, and no observations at all
of {\tt <=}.  How can that be?  What is happening
to ordered insertion?
\begin{tabbing}
{\tt hat-observe> insert }\\
{\tt \mkblue{1} insert 'p' \verb"_|_" = \verb"_|_" }\\
{\tt \mkblue{2} insert 'r' \verb"_|_" = \verb"_|_" }\\
{\tt \mkblue{3} insert 'o' \verb"_|_" = \verb"_|_" }\\
{\tt \mkblue{4} insert 'g' \verb"_|_" = \verb"_|_" }\\
{\tt \mkblue{5} insert 'a' \verb"_|_" = \verb"_|_" }\\
{\tt \mkblue{6} insert 'm' \verb"_|_" = \verb"_|_" }
\end{tabbing}
The symbol \verb"_|_" here indicates an undefined value.  Reading the character
arguments vertically {\tt "program"} seems to be misspelt: is there an observation
missing between {\tt 4} and {\tt 5}?  There are in fact two separate applications
{\tt insert 'r' \verb"_|_" = \verb"_|_"}, but duplicate observations are not listed (by default).

The {\tt insert} observations explain the fall in application counts.
In all the observed applications, the list arguments are undefined.
So neither of the defining equations for {\tt insert} is ever matched, 
there are no {\tt <=} comparisons (as these occur only in the right-hand
side of the second equation) and of course no recursive
calls.

Why are the {\tt insert} arguments undefined?  They should be the results of
{\tt sort} applications.
\begin{tabbing}
{\tt hat-observe> sort } \\
{\tt \mkblue{1} sort "program" = \verb"_|_" } \\
{\tt \mkblue{2} sort "rogram" = \verb"_|_" } \\
{\tt \mkblue{3} sort "ogram" = \verb"_|_" } \\
{\tt \mkblue{4} sort "gram" = \verb"_|_" } \\
{\tt \mkblue{5} sort "ram" = \verb"_|_" } \\
{\tt \mkblue{6} sort "am" = \verb"_|_" } \\
{\tt \mkblue{7} sort "m" = \verb"_|_" } \\
{\tt \mkblue{8} sort [] = \verb"_|_" }
\end{tabbing}
Observations {\tt 1} to {\tt 7} show applications of sort that
reduced to applications of {\tt insert}, with the {\tt \verb"_|_"} results
already observed\footnote{This insight requires independent knowledge
of the program, however,
as hat-observe, unlike hat-trail, is not concerned with
relationships between applications.}.
Observation {\tt 8} is the application that does not reduce.

In short, the story so far from hat-observe is quite simple: everything
is undefined!  What about the other two items in the
info list, {\tt putStrLn} and {\tt main}?
\begin{tabbing}
{\tt hat-observe> putStrLn } \\
{\tt \mkblue{1} putStrLn \verb"_|_" = \{IO\}} \\
{\tt hat-observe> main } \\
{\tt \mkblue{1} main = \{IO\}}
\end{tabbing}
Hat uses symbols enclosed in braces, such as {\tt \{IO\}} here,
to indicate values that cannot be shown more explicitly.
These observations only confirm that the program does compute
an I/O action, but the output string is undefined.

\subsection{Tracing a Non-terminating Computation}
Suppose we correct the first fault, by restoring the equation
\begin{tabbing}
{\tt sort [] = []}
\end{tabbing}
and recompile.  Now the result of running {\tt BadInsort} is a non-terminating
computation, with an infinite string {\tt aaaaaaa...} as output. It seems
that {\tt BadInsort} has entered an infinite loop. The computation can be
interrupted\footnote{When non-termination is suspected,
interrupt as quickly as possible to
avoid working with very large traces.}
by keying control-C.
\begin{tabbing}
{\tt \$ BadInsort } \\
{\tt Program interrupted.\ (\symbol{94}C)} \\
{\tt aaaaaaaaaaaaaaaaaaaaaaaaaaaaaaaaaaaaaaaaaaaaaaaaaaaaaaaaaaaaaaaaa\$ } \\
\end{tabbing}
Questions this time include:
\begin{itemize}
\item
What parts of the program does the infinite loop involve?
\item
How did it come about in the first place?
\end{itemize}

\subsubsection*{Using hat-trail}
The initial hat-trail display is:
\begin{tabbing}
{\tt Error:\ -------------------------------------------------------- } \\
\highlight{Program interrupted.\ (\symbol{94}C)} \\
{\tt Output:\ ------------------------------------------------------- } \\
{\tt aaaaaaaaaaaaaaaaaaaaaaaaaaaaaaaaaa... } \\
\end{tabbing}
We have a choice: we can follow the trail back either from the point
of interruption (the initial selection) or from the output (reached by
down-arrow).  In this case, it makes little difference\footnote{However,
the trace from point of interruption depends
on the timing of the interrupt.};
either way we
end up examining the endless list of {\tt 'a'}s.  Let's select the output:
\begin{tabbing}
{\tt Output:\ ------------------------------------------------------- } \\
\highlight{aaaaaaaaaaaaaaaaaaaaaaaaaaaaaaaaaa...} \\
{\tt Trail:\ -----------------  BadInsort.hs line:\ 7 col:\ 19 -------- } \\
{\tt <- putStrLn\ }\highlight{"aaaaaaaa..."} \\
{\tt <-\ }\highlight{insert 'p' ('a':\_) | if False}
\end{tabbing}
Notice two further features of expression display:
\begin{itemize}
\item
the ellipsis {\tt ...} in the string argument to {\tt putStrLn} indicates
the tail-end of a long string that has been pruned from the display
\footnote{In other contexts where large expressions have to be
pruned the symbol \CUT is used as a place-holder for components.};
\item
the symbol {\tt \verb"_"} in the list argument to {\tt insert} indicates an expression
that was never evaluated.
\end{itemize}
The parent application {\tt insert 'p' ('a':\verb"_") | if False} gives several
important clues.
It tells us that in the else-branch of the recursive case in the
definition of {\tt insert} the argument's head (here {\tt 'a'}) is duplicated
endlessly to generate the result without ever demanding the argument's
tail (shown only as \verb"_").  This should be enough explanation to discover
the fault if we didn't already know it.

\subsubsection*{Using hat-observe}
Once again, let's also see what happens if we use hat-observe.
\begin{tabbing}
\hspace{2.5cm}\=\hspace{2.5cm}\=\hspace{2.5cm}\=\hspace{2.5cm}\=\kill
{\tt hat-observe> :info } \\
{\tt 78 <=}       \> {\tt 1+83 insert}    \>  {\tt 1 main} \>
{\tt 1 putStrLn} \> {\tt 8 sort} \\
\end{tabbing}
All the expected items are listed as observable. The abnormal counts for
{\tt <=} and {\tt insert} already give a strong clue where to look.
But let's first consider the other functions.
We know well
enough from the overall output what {\tt main} and {\tt putStrLn} are doing,
but what about {\tt sort}? Its application count is perfect, but what can
we glean from details of the arguments and results?
\begin{tabbing}
{\tt hat-observe> sort } \\
{\tt \mkblue{1} sort "program" = "aaaaaaaaaa..." } \\
{\tt \mkblue{2} sort "rogram" = 'a':\verb"_" } \\
{\tt \mkblue{3} sort "ogram" = 'a':\verb"_" } \\
{\tt \mkblue{4} sort "gram" = 'a':\verb"_" } \\
{\tt \mkblue{5} sort "ram" = 'a':\verb"_" } \\
{\tt \mkblue{6} sort "am" = "a" } \\
{\tt \mkblue{7} sort "m" = "m" } \\
{\tt \mkblue{8} sort [] = [] }
\end{tabbing}
Observations 1 to 5 tell a similar story to hat-trail: the tails
of the recursively computed lists are never demanded; at the
outermost level, the head is repeated endlessly.
Observation 6 points to a problem other than non-termination,
but we shall ignore that for now.  Observations
7 and 8 do not point to a problem at all.

There is one further clue in these observations: the arguments decrease
just as expected, confirming that the recursive loop must be
in {\tt insert}.
\begin{tabbing}
{\tt hat-observe> insert } \\
{\tt \mkblue{1} insert 'p' ('a':\verb"_") = "aaaaaaaaaa..." } \\
{\tt \mkblue{2} insert 'r' ('a':\verb"_") = 'a':\verb"_" } \\
{\tt \mkblue{3} insert 'o' ('a':\verb"_") = 'a':\verb"_" } \\
{\tt \mkblue{4} insert 'g' ('a':\verb"_") = 'a':\verb"_" } \\
{\tt \mkblue{5} insert 'a' "m" = "a" } \\
{\tt \mkblue{6} insert 'm' [] = "m" } \\
{\tt searching ... (\symbol{94}C to interrupt) } \\
{\tt \{Interrupted\} }
\end{tabbing}
Many more observations would eventually be reported because
hat-observe lists each observation that is distinct from
% , or more
% general than\footnote{
% Application A is more general
% than application B if their arguments and results agree where both
% are evaluated but A's arguments are less evaluated than B's or A's
% result is more evaluated than B's.},
those listed previously.
When the computation is interrupted
there are many different applications of the form {\tt insert 'p' ('a':\verb"_")}
in progress, each with results evaluated to a different extent.

But observation 1 is enough. As the tail of the argument is
unevaluated, the result would be the same whatever the tail.
For example, it could be {\tt []}; so we know
{\tt insert 'p' "a" = "aaaa..."}.
This specific and simple
failing case directs us to the fault in the definition
of {\tt insert}.

\subsection{Tracing Wrong Output}

Let's now correct the recursive call from {\tt insert x (y:ys)} to
{\tt insert x ys}, recompile, then execute.
\begin{tabbing}
{\tt \$ BadInsort } \\
{\tt agop } \\
\end{tabbing}

\subsubsection*{Using hat-observe}
Once again, we could reach first for hat-trail to trace the fault,
but the availability of a well-defined (but wrong) result also
suggests a possible starting point in hat-observe:
\begin{tabbing}
{\tt hat-observe> insert \verb"_" \verb"_" = "agop" } \\
{\tt \mkblue{1} insert 'p' "agor" = "agop" }
\end{tabbing}
Somehow, insertion loses the final element {\tt 'r'}.
Perhaps we'd like to see more details of how this result is obtained ---
the relevant recursive calls, for example:
\begin{tabbing}
{\tt hat-observe> insert 'p' \verb"_" in insert } \\
{\tt \mkblue{1} insert 'p' "gor" = "gop" } \\
{\tt \mkblue{2} insert 'p' "or" = "op" } \\
{\tt \mkblue{3} insert 'p' "r" = "p" }
\end{tabbing}
Observation 3 makes it easy to discover the fault by inspection.

\subsubsection*{Using hat-trail}

If we instead use hat-trail, the same application could be reached
as follows. We first request the parent of the output; unsurprisingly
it is {\tt putStrLn "agop"}.  We then request the parent of the string
argument {\tt "agop"}:
\begin{tabbing}
{\tt Output:\ ------------------------------------------------------ } \\
\highlight{agop\char92n} \\
{\tt  } \\
{\tt Trail:\ ----------------- BadInsort.hs line:\ 10 col:\ 26 ------- } \\
{\tt <- putStrLn\ }\highlight{"agop"} \\
{\tt <-\ }\highlight{insert 'p' "agor" | if False} \\
\end{tabbing}
As in hat-observe, we see the {\tt insert} application that loses
the character {\tt 'r'}.
\begin{center}
{\large({\em To be continued.})}
\end{center}

\end{document}
